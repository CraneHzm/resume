% !TEX program = xelatex

\documentclass{resume}
%\usepackage{zh_CN-Adobefonts_external} % Simplified Chinese Support using external fonts (./fonts/zh_CN-Adobe/)
%\usepackage{zh_CN-Adobefonts_internal} % Simplified Chinese Support using system fonts

\begin{document}
\pagenumbering{gobble} % suppress displaying page number

\name{Yao (Marc) Wang}

\basicInfo{
  \email{yaowang95@pku.edu.cn} \textperiodcentered\ 
  \phone{(+86) 134-010-76331} \textperiodcentered\ 
  \faGithub \href{https://github.com/MarcWong/}{MarcWong}  
 }

\section{\faGraduationCap\ Education Experience}
\datedsubsection{\textbf{Peking University}, Beijing, China}{\textit{2017 -- Present}}
\textit{M.S.} in Computer Science, expected in July 2020, \quad GPA 3.44 / 4.0

\begin{itemize}
	\item \textbf{Relevant Courses:} \space
	Image and Video Based 3D Reconstruction, Computational Vision, Advanced Graphics Computing, Technique and Application of Deep Learning, Human Computer Interaction: Theory and Techniques
\end{itemize}


\datedsubsection{\textbf{Peking University}, Beijing, China}{\textit{2013 -- 2017}}
\textit{B.S.} in Intelligence Science and Technology, \quad GPA 3.34 / 4.0 (\textbf{top 30\%})

\begin{itemize}
	\item \textbf{Relevant Courses:} \space Introduction to Pattern Recognition, Digital Image Processing, Algorithm Design and Analysis, Introduction to Computer Systems, Data Structures and Algorithms, Set and Graph Theory, Web Software Technology
\end{itemize}

\section{\faFlask\ Research Experience}

\datedsubsection{\textbf{Peking University}, Beijing, China}{\textit{Sept. 2016 -- Present}}
\role{Student Researcher}{Manager : Yisong Chen}

\begin{itemize}
    \item Familiar with the theory of edge detection, multi-view stereo and semantic segmentation.
    \item Skilled in the workflow of semantic segmentation and depth estimation tasks, capable of efficiently providing pyTorch implementation contingent on the network structure diagram described in the paper
	\item Designed the protocol and labeled the urban drone semantic segmentation Dataset UDD
	\item Reduced the feature matching errors by segmentation information to enhance the performance of sparse reconstruction, which was published in \textit{PRCV 2018}
    \item Built a semantic reconstruction pipeline based on OpenMVS, R-MVSnet and Deeplab
\end{itemize}

\datedsubsection{\textbf{University of California, Los Angeles} California, USA}{\textit{Jun. 2016 - Sept. 2016}}
\role{Summer Intern}{Manager: Fabien Scalzo}

\begin{itemize}
  \item Became familiar with voxel construction and visualization method for blood flow in MR images
  \item Became familiar with the MATLAB image processing toolbox, especially image morphing and filtering
\end{itemize}


%%%%%%%%% Publications  %%%%%%%%%%%
\section{\faBookmark\ Publications}
\begin{itemize}
    \item \textbf{1$^{st}$ co$-$Author} Semantic 3D Reconstruction with Learning MVS and 2D Segmentation of Aerial Images[J] // Accepted by Mdpi's Applied Sciences, Special Issue "Augmented Reality, Virtual Reality \& Semantic 3D Reconstruction"
    \item \textbf{1$^{st}$ co$-$Author} Large-scale structure from motion with semantic constraints of aerial images[C]//Chinese Conference on Pattern Recognition and Computer Vision (PRCV). Springer, Cham, 2018: 347-359.
	\item \textbf{2$^{nd}$ Author} Sobel Heuristic Kernel for Aerial Semantic Segmentation[C]//2018 25$^{th}$ IEEE International Conference on Image Processing (ICIP). IEEE, 2018: 3074-3078.
	\item \textbf{1$^{st}$ Author} Abstract TP55: Spatio-temporal Flow Tractography (SFT) for Evaluation of Collateral Patterns in Acute Stroke[J]. 2017.
	\item \textbf{4$^{th}$ Author} Tensor Voting Extraction of Vessel Centerlines from Cerebral Angiograms. InInternational Symposium on Visual Computing 2016 Dec 12 (pp. 35-44). Springer, Cham.
\end{itemize}


%%%%%%%%% Honors  %%%%%%%%%%%
\section{\faHeartO\ Honors}

\datedsubsection{\textbf{2$^{nd}$ prize} in 3D Reconstruction Challenge Group, 
China Virtual Reality and Visualization Industry Technology Innovation Strategic Alliance}{\textit{Nov. 2019}}
\begin{itemize}
	\item The 2$^{nd}$ Virtual Reality Technology and Application Innovation Competition
\end{itemize}

\datedsubsection{\textbf{Merit Student}, Peking University}{\textit{2015, 2016, 2018}}
\begin{itemize}
	\item Schlumberger scholarship 2018
\end{itemize}

% \datedsubsection{\textbf{Excellent Grass-roots League Cadres}, Communist Youth League Beijing Committee}{\textit{2016}}
% \begin{itemize}
% 	\item Pioneer Cup Excellent Grass-roots League Cadres of Capital University and Secondary Vocational Colleges
% \end{itemize}

%%%%%%%%% Social Experience  %%%%%%%%%%%
% \section{\faUsers\ Social Experience}
% \datedsubsection{\textbf{Offpay Technology Co.,Ltd.}, Shenzhen, China}{\textit{July. 2018 -- present}}
% \role{CTO \& Frontend Engineer}

% \begin{itemize}
% 	\item Engaged in MVVM software developing pattern, practiced Vue.js in Wechat Miniprogram (mpvue).
% 	\item Made schedules and milestones for the 6-member fullstack team. Executed Scrum and Kanban agile development processes.
%     \item Miniprogram "Mian Dan Jun Shang Cheng" and "Mian Dan Jun Shang Hu" already ran online stablely for 2 mohths.
% \end{itemize}

% \datedsubsection{\textbf{Student Union of school of EECS}, Beijing, China}{\textit{Sept. 2015 -- Jun. 2016}}
% \role{Chairman}

% \begin{itemize}
% 	\item Founder of HackPKU, which attracted over 200 participants from PKU, THU, NYU and more universities.
% 	\item Applauded by the leaders of the college, it has been hosted for three times since and has become an iconic yearly event in our department.
%     \item The main sponsoring companies of the event include IBM, Microsoft, Sensetime, Nsfocus, QCloud, Umeng+ and Sohu.
% \end{itemize}

%%%%%%%%% Skills  %%%%%%%%%%%
\section{\faCogs\ Skills}

\begin{itemize}[parsep=0.5ex]
  \item Programming Languages: Python, MATLAB, C++, JavaScript, git, bash
  \item Languages: Mandarin (native), English (TOEFL 98), German (B1)
\end{itemize}

%%%%%%%%% Hobbies  %%%%%%%%%%%
% \section{\faHeartO\ Hobbies}

% \begin{itemize}[parsep=0.5ex]
%   \item Web/Miniprogram application building
%   \item Singing, chrous, drum
%   \item Hearthstone
% \end{itemize}


% \footnotetext[1]{https://github.com/MarcWong/semantic-point-cloud}
% \footnotetext[2]{http://mrright.wang/UDD}
% \footnotetext[3]{https://github.com/MarcWong/SemanticContour}
% \footnotetext[3]{https://github.com/MarcWong/PTI}


%% Reference
%\newpage
%\bibliographystyle{IEEETran}
%\bibliography{mycite}
\end{document}