% !TEX TS-program = xelatex
% !TEX encoding = UTF-8 Unicode
% !Mode:: "TeX:UTF-8"

\documentclass{resume}
\usepackage{zh_CN-Adobefonts_external} % Simplified Chinese Support using external fonts (./fonts/zh_CN-Adobe/)
%\usepackage{zh_CN-Adobefonts_internal} % Simplified Chinese Support using system fonts
\usepackage{linespacing_fix} % disable extra space before next section
\usepackage{cite}

\begin{document}
\pagenumbering{gobble} % suppress displaying page number
\name{\href{https://mrright.wang/}{王尧}}

\basicInfo{
  \email{yaowang95@pku.edu.cn} \textperiodcentered\ 
  \phone{(+86) 134-010-76331} \textperiodcentered\ 
  \faGithub \href{https://github.com/MarcWong/}{MarcWong} \textperiodcentered\ 
  \faGraduationCap \href{https://scholar.google.com/citations?user=X8je0QsAAAAJ}{Wang, Yao}
}
\section{\faGraduationCap\ 教育背景}
\datedsubsection{\textbf{北京大学}}{\textit{2017 -- 2020}}
\textit{硕士}\ 计算机软件与理论,总GPA 3.44
\begin{itemize}[parsep=0.5ex]
    \item 核心专业课程:深度学习技术与应用、现代图论、数字图像处理
\end{itemize}
\datedsubsection{\textbf{北京大学}}{\textit{2013 -- 2017}}
\textit{学士}\ 智能科学与技术,总GPA 3.34,排名8/35
\begin{itemize}[parsep=0.5ex]
	\item 核心课程:模式识别导论、数字图像处理、算法设计与分析、计算机系统导论、数据结构与算法、集合论与图论
\end{itemize}

\section{\faHeartO\ 获奖情况}

\datedsubsection{\textbf{三好学生},北京大学}{\textit{2014 - 2015, 2015 - 2016, 2017 - 2018}}
\begin{itemize}[parsep=0.5ex]
	\item 获2017 - 2018学年斯伦贝谢奖学金
\end{itemize}

\datedsubsection{\textbf{二等奖},
中国虚拟现实与可视化行业技术创新战略联盟,三维重建挑战组}{\textit{2019年9月}}
\begin{itemize}
	\item 第二届虚拟现实技术与应用创新大赛
\end{itemize}

\section{\faBookmark\ 发表论文}
\begin{itemize}
    \item \textbf{co$-$1$^{st}$ Author} Semantic 3D Reconstruction with Learning MVS and 2D Segmentation of Aerial Images. Appl. Sci. 2020, 10, 1275. 
    \item \textbf{co$-$1$^{st}$ Author} Large-scale structure from motion with semantic constraints of aerial images[C]//Chinese Conference on Pattern Recognition and Computer Vision (PRCV). Springer, Cham, 2018: 347-359.
	\item \textbf{2$^{nd}$ Author} Sobel Heuristic Kernel for Aerial Semantic Segmentation[C]//2018 25$^{th}$ IEEE International Conference on Image Processing (ICIP). IEEE, 2018: 3074-3078.
	\item \textbf{1$^{st}$ Author} Abstract TP55: Spatio-temporal Flow Tractography (SFT) for Evaluation of Collateral Patterns in Acute Stroke[J]. 2017.
	\item \textbf{4$^{th}$ Author} Tensor Voting Extraction of Vessel Centerlines from Cerebral Angiograms. In International Symposium on Visual Computing 2016 Dec 12 (pp. 35-44). Springer, Cham. 
\end{itemize}


\section{\faFlask\ 科研经历}

\datedsubsection{\textbf{北京大学}}{\textit{2016年9月 -- 2020年6月}}
\role{科研实习生}{主管: 陈毅松$ $副教授}
\begin{itemize}
    \item 理解三维立体视觉Structure from Motion和Multi-view Stereo的流程,掌握openMVS源码的深度图生成过程
    \item 理解二维语义分割任务的工作流,能够根据论文的网络结构图复现代码,掌握tensorflow和pytorch的使用
    \item 探索三维重建与语义分割任务的结合方式,实现三维点云K-近邻的语义稠密点云概率融合流程,并利用二维重投影语义图计算误差
	\item 主导标注无人机语义分割数据集UDD,利用图片二维语义对特征点匹配进行过滤,从而提升稀疏重建精度;相关工作投稿于PRCV 2018
\end{itemize}

\datedsubsection{\textbf{加州大学洛杉矶分校},  加利福尼亚, 美国}{\textit{2016年6月 - 9月}}
\role{科研实习生}{主管: Fabien Scalzo$ $副教授}
\begin{itemize}[parsep=0.5ex]
  \item 实现脑部核磁共振影像的数据处理及可视化,体素构建及血流量计算
  \item 熟练掌握MATLAB的图像处理函数(滤波器、形态学操作等)
\end{itemize}


%%%%%%%%% 社会经历  %%%%%%%%%%%
% \section{\faUsers\ 社会经历}

% \datedsubsection{\textbf{北京大学信息科学技术学院学生会},  }{2015年9月 - 2016年9月}
% \role{学生会主席}{}
% \begin{itemize}[parsep=0.5ex]
%   \item 创办北京大学黑客马拉松(HackPKU),吸引来自清华、NYU、哈工大等8所高校近200人参赛,获学院领导肯定,现已举办三届,成为信科学院的品牌活动
%   \item 赞助企业有商汤、IBM、微软、绿盟、青云、友盟+、搜狐、计蒜客等近10家
% \end{itemize}

% \datedsubsection{\textbf{免单君科技有限公司}, 深圳}{2018年7月 -- 2019年10月}
% \role{研发经理 \& 前端工程师}

% \begin{itemize}[parsep=0.5ex]
% 	\item 负责产品与技术团队的对接工作,管理5个人的前后端开发团队,掌握墨刀、看板等效率工具的使用
%     \item 熟练使用Vue.js(uniapp),“免单君商城”、“免单君商户”两个微信小程序均已平稳上线运营半年
% \end{itemize}

\section{\faCogs\ 技能}
\begin{itemize}[parsep=0.5ex]
  \item 编程语言: Python, MATLAB, JavaScript, C++, git
  \item 语言: 中文, 英语 (CET-6), 德语 (B1)
  
\end{itemize}

% \footnotetext[1]{https://github.com/MarcWong/semantic-point-cloud}
% \footnotetext[2]{http://mrright.wang/UDD}
% % \footnotetext[3]{https://github.com/MarcWong/SemanticContour}
% \footnotetext[3]{https://github.com/MarcWong/PTI}

%% Reference
%\newpage
%\bibliographystyle{IEEETran}
%\bibliography{mycite}
\end{document}
