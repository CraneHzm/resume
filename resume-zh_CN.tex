% !TEX TS-program = xelatex
% !TEX encoding = UTF-8 Unicode
% !Mode:: "TeX:UTF-8"

\documentclass{resume}
\usepackage{zh_CN-Adobefonts_external} % Simplified Chinese Support using external fonts (./fonts/zh_CN-Adobe/)
%\usepackage{zh_CN-Adobefonts_internal} % Simplified Chinese Support using system fonts
\usepackage{linespacing_fix} % disable extra space before next section
\usepackage{cite}

\begin{document}
\pagenumbering{gobble} % suppress displaying page number

\name{\href{https://mrright.wang/}{王尧}}

\basicInfo{
  \email{yaowang95@pku.edu.cn} \textperiodcentered\ 
  \phone{(+86) 134-010-76331} \textperiodcentered\ 
  \faGithub
  \href{https://github.com/MarcWong/}{MarcWong}  }
 
\section{\faGraduationCap\  教育背景}
\datedsubsection{\textbf{北京大学}}{2017 -- 至今}
\textit{硕士}\ 计算机软件与理论, 预计 2020 年 6 月毕业
\datedsubsection{\textbf{北京大学}}{2013 -- 2017}
\textit{学士}\ 智能科学与技术
\begin{itemize}[parsep=0.5ex]
	\item 总GPA 3.34
	\item 核心专业课程:数据结构与算法、计算机系统导论、数字图像处理、软件工程、人机交互
	\item 曾任信息科学技术学院学生会主席
	\item 创办首届北京大学黑客马拉松(HackPKU),吸引来自清华、上交、NYU、哈工大等8所高校近200人参赛,赞助企业有IBM、微软、商汤、绿盟、青云、友盟+、搜狐等10家
\end{itemize}
\datedsubsection{\textbf{人大附中}}{2010 -- 2013}
\datedsubsection{\textbf{清华附中}}{2007 -- 2010}

\section{\faHeartO\ 获奖情况}

\datedsubsection{\textbf{三好学生}, 北京大学}{2015, 2016, 2018}
\begin{itemize}[parsep=0.5ex]
	\item 获得2018学年斯伦贝谢奖学金
\end{itemize}

\datedsubsection{\textbf{优秀基层团干部}, 北京大学}{2017}
\begin{itemize}[parsep=0.5ex]
	\item 优秀基层团干部
\end{itemize}

\section{\faBookmark\ 发表论文}
\begin{itemize}[parsep=0.5ex]
	\item Hu T, \textbf{Wang Y}, Chen Y, et al. Sobel Heuristic Kernel for Aerial Semantic Segmentation[C]//2018 25th IEEE International Conference on Image Processing (ICIP). IEEE, 2018: 3074-3078.
	\item Chen Y, \textbf{Wang Y(共同一作)}, Lu P, et al. Large-scale structure from motion with semantic constraints of aerial images[C]//Chinese Conference on Pattern Recognition and Computer Vision (PRCV). Springer, Cham, 2018: 347-359.
	\item \textbf{Wang Y}, Shattuck D, Saver J, et al. Abstract TP55: Spatio-temporal Flow Tractography (SFT) for Evaluation of Collateral Patterns in Acute Stroke[J]. 2017.
	\item Ding Y, Nicolescu M, Farmer D, \textbf{Wang Y}, Bebis G, Scalzo F. Tensor Voting Extraction of Vessel Centerlines from Cerebral Angiograms. InInternational Symposium on Visual Computing 2016 Dec 12 (pp. 35-44). Springer, Cham. 
\end{itemize}


\section{\faFlask\ 科研经历}

\datedsubsection{\textbf{北京大学}}{2016年9月 -- 至今}
\role{科研实习生}{主管: 陈毅松 副教授}
\begin{itemize}[parsep=0.5ex]
	\item 对边缘检测及语义分割相关工作有较为全面的了解,为边缘检测网络HED设计后处理模块$^1$,实现非极大抑制(Non Maximum Suppression)方法,并设计了snake能量方程
	\item 设计了基于Sobel边缘检测算子的Sobel Heuristic Kernel(SHK),嵌在Resnet block后,优化语义边缘处的分割效果,取得了Inria Aerial Image Labeling Dataset竞赛$^2$的第二名(2018年2月), 相关工作投稿于ICIP 2018
	\item 整理标注了语义分割数据集UDD$^3$,探索二维语义信息在三维重建中的作用, 相关工作投稿于PRCV 2018
\end{itemize}

\datedsubsection{\textbf{加州大学洛杉矶分校},  加利福尼亚, 美国}{2016年6月 - 9月}
\role{科研实习生}{主管: Fabien Scalzo 副教授}
\begin{itemize}[parsep=0.5ex]
  \item 脑部核磁共振影像的数据处理及可视化,体素构建及血流量计算$^4$
  \item 熟练掌握MATLAB的图像处理函数(滤波器、形态学操作等)
\end{itemize}

% Reference Test
%\datedsubsection{\textbf{Paper Title\cite{zaharia2012resilient}}}{May. 2015}
%An xxx optimized for xxx\cite{verma2015large}
%\begin{itemize}
%  \item main contribution
%\end{itemize}


\section{\faUsers\ 企业经历}
\datedsubsection{\textbf{免单君科技有限公司}, 深圳}{\textit{2018年7月 -- 至今}}
\role{CTO \& 前端工程师}

\begin{itemize}
	\item 熟悉软件设计范式MVVM、MVC,利用Vue.js 开发微信小程序(mpvue)
	\item 熟悉Scrum敏捷开发流程,为6个人的全栈开发团队制定开发计划,

\end{itemize}

\section{\faCogs\ 技能}
% increase linespacing [parsep=0.5ex]
\begin{itemize}[parsep=0.5ex]
  \item 编程语言: JS > MATLAB = C++ > Python
% \end{itemize}
% \section{\faInfo\ 其他}
% increase linespacing [parsep=0.5ex]
% \begin{itemize}[parsep=0.5ex]
  \item 爱好: 唱歌、足球、台球
  \item 语言: 中文 (母语), 英语 (流利), 德语 (B1)
\end{itemize}


\footnotetext[1]{https://github.com/MarcWong/SemanticContour}
\footnotetext[2]{https://project.inria.fr/aerialimagelabeling/leaderboard/}
\footnotetext[3]{http://mrright.wang/UDD}
\footnotetext[4]{https://github.com/MarcWong/PTI}

%% Reference
%\newpage
%\bibliographystyle{IEEETran}
%\bibliography{mycite}
\end{document}
